\documentclass[a4]{article}
\usepackage[utf8]{inputenc}
\usepackage[french]{babel}
\usepackage{listings}
\usepackage{color}
\usepackage{graphicx}
\usepackage[T1]{fontenc}
\usepackage{pdfpages}
\usepackage{geometry}
\geometry{hmargin=2.5cm,vmargin=2.5cm}

\definecolor{mygreen}{rgb}{0,0.6,0}
\definecolor{mygray}{rgb}{0.5,0.5,0.5}
\definecolor{mymauve}{rgb}{0.58,0,0.82}

\lstset{
  backgroundcolor=\color{white},   % choose the background color; you must add \usepackage{color} or \usepackage{xcolor}
  basicstyle=\footnotesize,        % the size of the fonts that are used for the code
  breakatwhitespace=false,         % sets if automatic breaks should only happen at whitespace
  breaklines=true,                 % sets automatic line breaking
  captionpos=b,                    % sets the caption-position to bottom
  commentstyle=\color{mygreen},    % comment style
  deletekeywords={...},            % if you want to delete keywords from the given language
  escapeinside={\%*}{*)},          % if you want to add LaTeX within your code
  extendedchars=true,              % lets you use non-ASCII characters; for 8-bits encodings only, does not work with UTF-8
  frame=L,	                       % adds a frame around the code
  keepspaces=true,                 % keeps spaces in text, useful for keeping indentation of code (possibly needs columns=flexible)
  keywordstyle=\color{blue},       % keyword style
  language=C,                 	   % the language of the code
  otherkeywords={*,...},           % if you want to add more keywords to the set
  numbers=none,                    % where to put the line-numbers; possible values are (none, left, right)
  numbersep=5pt,                   % how far the line-numbers are from the code
  numberstyle=\tiny\color{mygray}, % the style that is used for the line-numbers
  rulecolor=\color{black},         % if not set, the frame-color may be changed on line-breaks within not-black text (e.g. comments (green here))
  showspaces=false,                % show spaces everywhere adding particular underscores; it overrides 'showstringspaces'
  showstringspaces=false,          % underline spaces within strings only
  showtabs=false,                  % show tabs within strings adding particular underscores
  stepnumber=2,                    % the step between two line-numbers. If it's 1, each line will be numbered
  stringstyle=\color{mymauve},     % string literal style
  tabsize=2,	                   % sets default tabsize to 2 spaces
  title=\lstname                   % show the filename of files included with \lstinputlisting; also try caption= instead of title
}
%gestion des caractères latins
\lstset{literate=
  {á}{{\'a}}1 {é}{{\'e}}1 {í}{{\'i}}1 {ó}{{\'o}}1 {ú}{{\'u}}1
  {Á}{{\'A}}1 {É}{{\'E}}1 {Í}{{\'I}}1 {Ó}{{\'O}}1 {Ú}{{\'U}}1
  {à}{{\`a}}1 {è}{{\`e}}1 {ì}{{\`i}}1 {ò}{{\`o}}1 {ù}{{\`u}}1
  {À}{{\`A}}1 {È}{{\'E}}1 {Ì}{{\`I}}1 {Ò}{{\`O}}1 {Ù}{{\`U}}1
  {ä}{{\"a}}1 {ë}{{\"e}}1 {ï}{{\"i}}1 {ö}{{\"o}}1 {ü}{{\"u}}1
  {Ä}{{\"A}}1 {Ë}{{\"E}}1 {Ï}{{\"I}}1 {Ö}{{\"O}}1 {Ü}{{\"U}}1
  {â}{{\^a}}1 {ê}{{\^e}}1 {î}{{\^i}}1 {ô}{{\^o}}1 {û}{{\^u}}1
  {Â}{{\^A}}1 {Ê}{{\^E}}1 {Î}{{\^I}}1 {Ô}{{\^O}}1 {Û}{{\^U}}1
  {œ}{{\oe}}1 {Œ}{{\OE}}1 {æ}{{\ae}}1 {Æ}{{\AE}}1 {ß}{{\ss}}1
  {ű}{{\H{u}}}1 {Ű}{{\H{U}}}1 {ő}{{\H{o}}}1 {Ő}{{\H{O}}}1
  {ç}{{\c c}}1 {Ç}{{\c C}}1 {ø}{{\o}}1 {å}{{\r a}}1 {Å}{{\r A}}1
  {€}{{\EUR}}1 {£}{{\pounds}}1
}
%definition d'un syle pour les documents text
\lstdefinestyle{txt}{
	frame=none,
	numbers=none,
	stringstyle=\color{black},
}

\begin{document}
	\title{\Huge{\textbf{Rapport TER}}}
	\author{Alabi Steve - El Harti Zakaria - Chouipe Thibault \\ \\ \\ 
		Dirigé par Mr Mautor \\ \\ \\ \\
		Developpement d’une méthode de recherche arborescente pour \\ un jeu à 2 joueurs : application au 
jeu Gobblet	\\ \\ \\}
	\date{29 Mars 2018}
		

	\begin{titlepage}
		\maketitle
		\vspace{20em}
	\end{titlepage}
	
	\section{Introduction}
	
	-Rappel sujet et but projet(nom projet, prof, année, matière, nom sujet..) \\
	- \\
	
	
	
	
	\section{Sommaire}
	
	ICI PLAN \\ \\
	
	
	
	
	
	\section{Gobblet, le jeu}
		\subsection{Description du jeu}
				Règles, petite image du jeu, trouvable sur internet..etc, origine\\ \\
	
		\subsection{Explication du problème}
				-IA ...etc\\
				-Fonction d'evaluation\\ \\
				La fonction d'évaluation attribue un score a chaque feuille (qui représente l'état du jeu a un instant t).
Plus le score est élever, plus l'état de jeu est favorable a l'IA.
Fonction d'évaluation :
	Séparé en 2 catégorie : Bonus/Malus
	Malus:
		Si 2 pions adverses sur la même ligne (ou grandes diagonales) = -50
		Si 3 pions adverses sur la même ligne (ou grandes diagonales)= -100
		Si 3 pions adverses sur la même ligne (ou grandes diagonales)= -150
	Bonus:
		Si 2 pions amis sur la même ligne (ou grandes diagonales)= +50
		Si 3 pions amis sur la même ligne (ou grandes diagonales)= +100
		Si 3 pions amis sur la même ligne (ou grandes diagonales)= +150
			A partir d'un alignemement superieur a 2:
					Pour chaque tres gros : +5
					Pour chaque gros : +4
					Pour chaque moyen : +3
					Pour chaque petit : +2
rq : en additionnant les bonus du au type de pièces alignés, on ne doit pas dépasser 50 (car on considère qu'il est 
plus important d'agrandir sa ligne plutot que de la renforcer).
	\section{Partie code}
					\subsubsection{Algo et implementation}
					
					\subsubsection{Algo minimax}
Principe :\\
L'objectif est de choisir le meilleur coup M a jouer pour l'IA.\\ \\

L'algorithme minimax (aussi appelé algorithme MinMax) est un algorithme qui s'applique
 à la théorie des jeux pour les jeux à deux joueurs à somme nulle (et à information complète)
  consistant à minimiser la perte maximum (c'est-à-dire dans le pire des cas). \\ \\


Exemple sur une prise de décisions entre deux coups différents avec une profondeur de 2.
Pour ce faire, si l'IA a le choix entre deux coups B1 et B2, on va regarder ce que l'humain peut jouer
après chacun de ces deux coups. Soit B1.a et B1.b et B2.a et B2.b les coups suivants, on va affecter a B1
 et B2 la note minimale
du coup suivant (Soit B1 = min(B1.a;B1.b) et B2 = min(B2.a; B2.b). Pourquoi le minimum ? Car on va considérer que 
l'humain joue de facon optimale. Il va donc chercher et jouer le coup qui l'amène vers la victoire (et donc le plus mauvais 
coup pour l'IA).
Ensuite, l'IA va choisir le maximum entre B1 et B2(Soit M = max(B1,B2) car c'est le coup qui rapporte le plus tout en prenant
en compte le
fait que l'adversaire ne pourra pas jouer un "trop" mauvais coup qui le mettrait dans une situation compliquée le tour
suivant.
										%INTEGRER IMAGE 1 schema	
											\subsubsection{Algo alpha beta}
Élagage alpha-bêta
Principe:

C'est une technique permettant de réduire le nombre de noeuds évalués par l'algorithme minimax.

L'algorithme minimax effectue en effet une exploration complète de l'arbre de recherche jusqu'à un niveau donné,
alors qu'une exploration partielle de l'arbre est généralement suffisante : lors de l'exploration, il n'est pas nécessaire
d'examiner les sous-arbres qui conduisent à des configurations dont la valeur ne contribuera pas au calcul du gain
à la racine de l'arbre.
Exemple :
										%INTEGRER IMAGE 2
										
										
Le nœud MIN vient de mettre à jour sa valeur courante à 4. Celle-ci, 
qui ne peut que baisser, est déjà inférieure à alpha=5, la valeur actuelle du noeud MAX précédent.
 Celui-ci cherchant la valeur la plus grande possible, ne la choisira donc de toute façon pas
 
 Le noeud MIN vient de mettre à jour sa valeur courante à 6. Celle-ci, qui ne peut que baisser, est déjà égale à alpha=6, la valeur
  actuelle du nœud MAX précédent. Celui-ci cherchant une valeur supérieure, il ne mettra de toute façon pas à jour sa valeur
   que ce nœud vaille 6 ou moins.
	
	
Le nœud MIN vient de mettre à jour sa valeur courante à 5. Celle-ci, qui ne peut que baisser, 
est déjà inférieure à alpha=6, la valeur actuelle du nœud MAX précédent. Celui-ci cherchant la valeur la plus grande possible,
 ne la choisira donc de toute façon pas.	
										
										
					\subsubsection{Choix langage, bibliothèque,}	
					   Java..etc
					\subsubsection{Fonctionnement du jeu au niveau du code, details de qques fonctions}

	\section{Simulation d'une partie, de coups joués}
	-Image..etc
	
	\section{Conclusion}
	
	\section{Bibliographie}
		

	\section{Annexe}

	
\end{document}
